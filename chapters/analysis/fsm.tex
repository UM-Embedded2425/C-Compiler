\section{FSM}
\shortcut{analysis}{fsm}

\vspace{0.75cm}


In the context of the 8051 microcontroller project, a \textbf{Finite State Machine (FSM)} 
is a key design structure used to manage the control flow of sequential logic elements,
particularly in the CPU and peripheral modules. FSMs are essential for defining how a 
system transitions between various operational states based on inputs and internal conditions.

\subsection*{1. CPU Control Unit}

In both the single-cycle and pipelined implementations of the 8051 CPU, FSMs are employed
in the control unit to govern the sequence of operations. The instruction execution path 
is typically broken into the following stages:

\begin{enumerate}
    \item \textbf{Fetch (IF)} – Instruction is fetched from instruction memory.
    \item \textbf{Decode (ID)} – Instruction is decoded, and control signals are prepared.
    \item \textbf{Memory Access (MEM)} – Any required memory read or write is performed.
    \item \textbf{Execute (EX)} – The ALU performs computation or address resolution.
    \item \textbf{Write-back (WB)} – Results are written back to the register file.
\end{enumerate}

The FSM transitions between these states on each clock cycle, generating appropriate control 
signals to activate the required components. This staged breakdown allows the design to 
support more complex operations and prepares the structure for pipelining.

\subsection*{2. Peripherals}

Each peripheral device (e.g., UART, timer, GPIO) includes its own FSM to manage internal
behavior such as data transfer, mode switching, and status updates. For example, a UART FSM
might transition through states such as 
\texttt{Idle}, \texttt{Start}, \texttt{Data Transfer}, \texttt{Stop}, and \texttt{Ready}.

\subsection*{3. Compiler Integration and Memory Loader}

FSMs can also be used in modules that handle:
\begin{itemize}
    \item Instruction or data memory loading prior to program execution.
    \item Interaction with a C compiler backend or debugger interface.
\end{itemize}

These FSMs typically control the sequencing of memory writes, flag signaling, or program
counter initialization.

\subsection*{Summary}

FSMs offer a reliable, modular, and synthesize approach to managing sequential behavior 
in both CPU and peripheral subsystems. Their deterministic nature ensures predictable 
transitions and clear operational flow, which is essential for the complexity of a 
multistage microcontroller like the 8051.
